\chapter{Introduction}

\section{Overview}

\section{Aims and Objectives}
\subsection{Primary}

\begin{itemize}
\item Create a web application
\item The application must allow the user to write programs
\item The writing of these programs will be done straight into a visual representation of the language's abstract syntax tree
\item The programs must be able to be run within the application
\end{itemize}

\subsection{Secondary}

\begin{itemize}
\item The execution of these programs must be shown visually with relevant feedback
\subitem e.g. showing the steps of evaluating an expression
\item The execution must be able to be paused and rewound at any time
\item Code changes must be able to be done at any time - Mostly done
\item There will be some way for the programs to have input and output
\item There will be multiple types of possible program input and output
\subitem e.g. drawing on a canvas, inputting/outputting tabular data
\item There will be several different programming concepts/styles supported
\subitem e.g. functional, scripting, object-oriented, bare-metal
\item Multiple tweaked versions of the program will be able to run in parallel to visually demonstrate algorithmic differences between them
\end{itemize}
\subsection{Tertiary}

\begin{itemize}
\item Multiple users will be able to work collaboratively on programs
\item It will be possible to export the program to a professional and compatible programming tool
\subitem - e.g. export to Python for scripted, Java for object-oriented, Haskell for functional
\item The visual representation of the code can be changed to represent higher-level concepts
\subitem e.g. object-oriented to UML
\end{itemize}

\section{Key aspects}