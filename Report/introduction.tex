\chapter{Introduction}

\section{Overview}

The aim of this project is to create a visual programming environment for novices. This tool allows programs to be written, modified and executed in a graphical environment designed from the ground up for novices. Unlike normal programming education for novices, programs are specified in toy languages designed specifically to teach the fundamentals of computation. It is not possible to use this tool to create real programs that would perform real tasks; allowing this would dilute the purpose and suitability of this tool for novices while not benefiting more experienced programmers.

Overall this project was successful and I was able to implement a significant proportion of the initial aims and objectives. Many of those not completed would be straightforward to add with minimal extra work. I believe that there is much more work that can be done to improve the tool for its target audience, much of in areas which I did not know at the beginning of this project.

I believe that the implementation of the virtual machine is a particular highlight of this project. This is an element which I did not initially consider necessary but in retrospect, it was the right tool for executing the user programs.

During the design and implementation phase of this project, it also became clear that there is significant potential to use this tool or technologies within for further computer science education. This potential is discussed at length in the evaluation section.

Instructions for building and executing this software are included as an appendix. An important goal for this tool was for the implementation to rely solely on standard web technologies, rather than being tied to any one platform. As a result, this software is presented as a static web page which can be run in any modern web browser. The compilation process requires basic web development tooling, but these are also platform-independent and open source.

\section{Aims and Objectives}
\subsection{Primary}

\begin{itemize}
\item Create a web application
\item The application must allow the user to write programs
\item The writing of these programs will be done straight into a visual representation of the language's abstract syntax tree
\item The programs must be able to be run within the application
\end{itemize}

\subsection{Secondary}

\begin{itemize}
\item The execution of these programs must be shown visually with relevant feedback
\subitem e.g. showing the steps of evaluating an expression
\item The execution must be able to be paused and rewound at any time
\item Code changes must be able to be done at any time - Mostly done
\item There will be some way for the programs to have input and output
\item There will be multiple types of possible program input and output
\subitem e.g. drawing on a canvas, inputting/outputting tabular data
\item There will be several different programming concepts/styles supported
\subitem e.g. functional, scripting, object-oriented, bare-metal
\item Multiple tweaked versions of the program will be able to run in parallel to visually demonstrate algorithmic differences between them
\end{itemize}
\subsection{Tertiary}

\begin{itemize}
\item Multiple users will be able to work collaboratively on programs
\item It will be possible to export the program to a professional and compatible programming tool
\subitem - e.g. export to Python for scripted, Java for object-oriented, Haskell for functional
\item The visual representation of the code can be changed to represent higher-level concepts
\subitem e.g. object-oriented to UML
\end{itemize}